\documentclass{beamer}

\usepackage[english]{babel}
\usepackage[T1]{fontenc}
\usepackage{graphicx}
\usepackage{moresize}
\usepackage{mathptmx}
\usepackage{helvet}
\usepackage{hyperref}
\usepackage[final]{pdfpages}
\usepackage{tcolorbox}

\tcbuselibrary{skins}
\usefonttheme{serif}

%\usetheme{Madrid}

\definecolor{th_black}{rgb}{0.0, 0.051, 0.075}
\definecolor{th_red}{rgb}{0.8, 0.176, 0.196}
\definecolor{th_orange}{rgb}{0.827, 0.373, 0.161}
\definecolor{th_purple}{rgb}{0.616, 0.192, 0.549}
\definecolor{light_gray}{rgb}{0.741, 0.784, 0.812}
\definecolor{dark_gray}{rgb}{0.439, 0.475, 0.498}
\definecolor{title_box}{HTML}{b683ca}%{a767bf}

\newtcolorbox{remark_box}[1][Theorem:]{
beamer,
colback=white,
colbacktitle=white,
coltitle=th_purple,
colframe=th_orange,
boxrule=1pt,
titlerule=0pt,
arc=10pt,
title={\strut#1}
}

\newtcolorbox{attention_box}[1][Theorem:]{
beamer,
colback=white,
colbacktitle=white,
coltitle=th_red,
colframe=th_orange,
boxrule=1pt,
titlerule=0pt,
arc=10pt,
title={\strut#1}
}

\setbeamerfont{title}{size=\HUGE}
\setbeamerfont{frametitle}{size=\Large, series=\bfseries}

\setbeamercolor{title}{bg=white, fg=dark_gray}
\setbeamercolor{frametitle}{bg=white, fg=dark_gray}

\setbeamercolor{section number projected}{bg=th_red}
\setbeamercolor{subsection number projected}{bg=th_purple}
\setbeamercolor{section in toc}{fg=black}
\setbeamercolor{subsection in toc}{fg=darkgray}

\setbeamercolor{header left}{bg=th_red}
\setbeamercolor{header center}{bg=th_orange}
\setbeamercolor{header right}{bg=th_purple}

\setbeamercolor{enumerate item}{fg=th_purple}


\setbeamertemplate{itemize item}{\color{th_red}$\blacktriangleright$}
\setbeamertemplate{itemize subitem}{\color{th_purple}$\blacktriangleright$}
%\setbeamertemplate{itemize subsubitem}{\color{orange}$\blacktriangleright$}

\setbeamertemplate{title page}[default][left]%colsep=-4bp,rounded=true, shadow=true

\setbeamertemplate{section in toc}[ball unnumbered]
\setbeamertemplate{subsection in toc}[ball unnumbered]

\beamertemplatenavigationsymbolsempty

\setbeamertemplate{footline}{%
	\leavevmode
    \begin{beamercolorbox}[wd=0.125\paperwidth,dp=1pt]{}
    \end{beamercolorbox}%
    \begin{beamercolorbox}[wd=0.875\paperwidth]{}
    \hrule
    \vspace{0.1mm}
    \hrule
    \vspace{1mm}
    \parbox[b]{0.5\paperwidth}{\inserttitle\\[1.5mm] \insertshortauthor\\ \insertshortdate}
    \hfill
    Slide~\insertframenumber~of~\inserttotalframenumber
    \hspace{5mm}
    \includegraphics[width=0.12\paperwidth]{sources/logo_TH-Koeln_CMYK_22pt}
    \hspace{2mm}
    \vspace{1mm}
    \end{beamercolorbox}%
}

\setbeamertemplate{headline}{%
    \leavevmode
    \begin{beamercolorbox}[wd=0.125\paperwidth,ht=1pt,dp=4pt]{}
    \end{beamercolorbox}%
    \begin{beamercolorbox}[wd=0.292\paperwidth,ht=1pt,dp=4pt]{header left}
    \end{beamercolorbox}%
    \begin{beamercolorbox}[wd=0.292\paperwidth,ht=1pt,dp=4pt]{header center}
    \end{beamercolorbox}%
    \begin{beamercolorbox}[wd=0.292\paperwidth,ht=1pt,dp=4pt]{header right}
    \end{beamercolorbox}
} 


%Information to be included in the title page:
\title{Twitter Bot}
\subtitle{Final Presentation}
\author[Sina Behdidka, Tim Mennicken, Robert Rose]{Sina Behdidka\\ Tim Mennicken\\ Robert Rose }
\institute[TH Köln]{University of Applied Sciences Cologne}
\date[04.02.2020] % (optional)
{Tuesday the 4th of February, 2020}
%\logo{\includegraphics[height=0.8cm]{sources/twitter.png}}
 
\begin{document}
\bgroup
\makeatletter
\setbeamertemplate{footline}
{
	\leavevmode
    \begin{beamercolorbox}[wd=0.125\paperwidth,dp=1pt]{}
    \end{beamercolorbox}%
    \begin{beamercolorbox}[wd=0.875\paperwidth,dp=0ex]{}
    \hrule
    \vspace{0.1mm}
    \hrule
    \vspace{1mm}
    \parbox[b]{0.3\paperwidth}{\inserttitle\\[1.5mm] \insertshortauthor\\ \insertshortdate}
    \hfill
    \includegraphics[width=0.12\paperwidth]{sources/logo_TH-Koeln_CMYK_22pt}
    \hspace{2mm}
    \vspace{1mm}
    \end{beamercolorbox}%
}
\makeatother
\begin{frame}
\titlepage
\end{frame}
\egroup

\setcounter{framenumber}{0}

\begin{frame}
\frametitle{Table of Contents}
%\begin{multicols}{2}
\tableofcontents
%\end{multicols}
\end{frame} 
 
\section{Introduction}

\section{Methods}

\section{Data Acquisition}

\begin{frame}{Data Acquisition}
\begin{columns}
\column{0.5\textwidth}
\begin{itemize}
\item Twitter developer platform
	\begin{itemize}
	\item Twitter developer account needed
	\item Tweepy: Python wrapper
	\item Complete functionality
	\end{itemize}
\end{itemize}
\vspace{0.5cm}
\centering{\includegraphics[height=2.5cm]{sources/twitter.png}}
\pause
\column{0.5\textwidth}
\begin{itemize}
\item GetOldTweets
	\begin{itemize}
	\item Fetches Tweets from the website
	\item Python package
	\item Read only access
	\end{itemize}
\end{itemize}
\vspace{0.5cm}
\centering{\includegraphics[height=2.5cm]{sources/pypi.jpg}}
\end{columns}
\end{frame}

\subsection{Twitter API}

\begin{frame}{Data Acquisition}
\framesubtitle{Twitter API}
\begin{itemize}
\item Amount of Tweets limited
	\begin{itemize}
	\item Request windows are separated in 15 minutes chunks
	\item Specific amount of requests per window
	\item Can be bypassed by cyclic requests paired with pauses
	\end{itemize}
\vspace{5mm}
\item<2-> Cannot go arbitrarily far to the past
	\begin{itemize}
	\item Returns no Tweets older than roundabout a month
	\item Cannot be bypassed
	\end{itemize}
\vspace{5mm}
\item<3-> Account got blacklisted
	\begin{itemize}
	\item No further access to the Twitter API
	\item Got unblocked on request
	\end{itemize}
\end{itemize}
\end{frame}

\subsection{GetOldTweets}

\begin{frame}{Data Acquisition}
\framesubtitle{GetOldTweets}
\begin{itemize}
\item No limitation
	\begin{itemize}
	\item Arbitrary amount of Tweets
	\item No restrictions with respect to publication date
	\end{itemize}
\vspace{5mm}
\item<2-> Shortened functionality
	\begin{itemize}
	\item Limited meta data of Tweets
	\item No functionality for publishing Tweets
	\end{itemize}
\vspace{5mm}
\item<3-> Better suited for getting large data sets
\end{itemize}
\end{frame}

\section{Pre-Processing}

\begin{frame}{Pre-Processing}
\framesubtitle{Particular Content}
\begin{itemize}
\item Weblinks
\item Picture \& video links
\item Punctuation symbols
\item Special characters
\item Retweets
\item Hashtags
\item Username references
\end{itemize}
\end{frame}

\begin{frame}{Pre-Processing}
\framesubtitle{Particular Content}
\begin{remark_box}[Content we keep:]
\begin{enumerate}
 \item Selected punctuation symbols
 \item[] \small{point, comma, exclamation mark, interrogation mark, colon and hash}
 \normalsize
 \item Hashtags
 \item References to usernames
\end{enumerate}
\end{remark_box} 
\end{frame}

\begin{frame}{Pre-Processing}
\framesubtitle{Space characters}
\begin{itemize}
\item Union of several space characters
\item The Tokenizer splits the input text at space characters
\end{itemize}
\vspace{5mm}
\pause
\begin{attention_box}[Why we need to add spaces]
house.~~$\Rightarrow$ [\textquotedbl house.\textquotedbl]\\
house .~$\Rightarrow$ [\textquotedbl house\textquotedbl , \textquotedbl .\textquotedbl]
\end{attention_box}
\end{frame}

\begin{frame}{Pre-Processing}
\framesubtitle{Termination Symbol}
\end{frame}

\section{Software}

\begin{frame}{Software}
\framesubtitle{Structure}
\end{frame}

\begin{frame}{Software}
\framesubtitle{Interaction}
\end{frame}

\section{Experiments}

\section{Post-Processing}

\section{References}

\begin{frame}{References}
\begin{small}
\begin{itemize}
\item Twitter Developers -- \url{https://developer.twitter.com/en.html}
\item GetOldTweets -- \url{https://github.com/Jefferson-Henrique/GetOldTweets-python}
\end{itemize}
\end{small}
\end{frame}

%\begin{thebibliography}{00}
%\bibitem{tim1} O. Varol, ``Online Human-Bot Interactions: Detection, Estimation and Characterization''. \textit{ICWSM'17}, Montreal, 2017.
%\bibitem{tim2} S. Kudugunta and E. Ferrara, ``Deep Neural Network for Bot Detection''. \textit{Information Sciences}, vol. 467, pp. 312-322, 2018.
%\bibitem{tim3} N. Michael, ``Towardsdatascience'', 2018 [Online]. Available: \url{https://towardsdatascience.com/illustrated-guide-to-lstms-and-gru-s-a-step-by-step-explanation-44e9eb85bf21}. [Accessed 22- Jan- 2020].
%\bibitem{tim4} Tensorflow, 2020 [Online]. Available: \url{https://www.tensorflow.org/tutorials/text/word_embeddings#using_the_embedding_layer}. [Accessed 22- Jan- 2020].
%\bibitem{tim5} L.A. dos Santos, 2018 [Online]. Available: \url{https://leonardoaraujosantos.gitbooks.io/artificial-inteligence/content/dropout_layer.html}. [Accessed 22- Jan- 2020].
%\bibitem{twidev} Twitter Developers, [Online]. Available: \url{https://developer.twitter.com/en.html}. [Accessed: 21- Jan- 2020].
%\bibitem{got} Jefferson Henrique, ``GetOldTweets-python - A project written in Python to get old tweets'' [Online]. Available: \url{https://github.com/Jefferson-Henrique/GetOldTweets-python}. [Accessed: 21- Jan- 2020].
%\bibitem{tweepy} Aaron Hill, Joshua Rosslein and Harmon, ``Tweepy - Twitter for Python'' [Online]. Available: \url{https://github.com/tweepy/tweepy}. [Accessed: 21- Jan- 2020].
%\bibitem{nltk} NLTK Project, ``Natural Language Toolkit 3.4.5'' [Online]. Available: \url{https://www.nltk.org/}. [Accessed: 21- Jan- 2020].
%\bibitem{twidev_rates} Twitter Developers, ``Rate limits'' [Online]. Available: \url{https://developer.twitter.com/en/docs/basics/rate-limits}. [Accessed: 25- Jan- 2020].
%\bibitem{tutrob} Jason Brownlee, ``How to Develop a Word-Level Neural Language Model and Use it to Generate Text'' [Online]. Available: \url{https://machinelearningmastery.com/how-to-develop-a-word-level-neural-language-model-in-keras/}. [Accessed: 22- Jan- 2020].
%\bibitem{attention} Asish Vaswani, Noam Shazeer, Niki Parmer, Jakob Uszkoreit, Llion Jones, Aidan N. Gomez, Lukasz Kaiser, Illia Polosukhin, ``Attention Is All You Need'' 2017 [Online]. Available: \url{https://arxiv.org/abs/1706.03762}. [Accessed: 22- Jan- 2020].
%\end{thebibliography}

\end{document}

